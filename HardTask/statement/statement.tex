\documentclass[11pt,a4paper]{article}

\usepackage{indentfirst}
\usepackage{amssymb}
\usepackage{subcaption}
\usepackage{graphicx}
\usepackage{longtable}
\usepackage{fancyhdr}
\usepackage{xeCJK}
\usepackage{amsmath}
\usepackage{amssymb}
\usepackage{ulem}
\usepackage{xcolor}
\usepackage{fancyvrb}
\usepackage{listings}
\usepackage{soul}
\usepackage{hyperref}
	\lstdefinestyle{C}{
   		language=C, 
   		basicstyle=\ttfamily\bfseries,
    	numbers=left, 
    	numbersep=5pt,
    	tabsize=4,
    	frame=single,
   	 	commentstyle=\itshape\color{brown},
    	keywordstyle=\bfseries\color{blue},
   	 	deletekeywords={define},
    	morekeywords={NULL,bool}
	}	

\setCJKmainfont{Noto Sans Mono CJK TC}
 
\voffset -20pt
\textwidth 410pt
\textheight 650pt
\oddsidemargin 20pt
\newcommand{\XOR}{\otimes}
\linespread{1.2}\selectfont
\graphicspath{{images/}}

\pagestyle{fancy}
\lhead{國立宜蘭高中 111 學年度資訊學科能力競賽 - 測機}

\begin{document}

\begin{center}
\section*{C. Hard Task}
\end{center}

\section*{Description}

這是一題非常困難的測試題,你會拿到一個數字 $K$,請構造三個介於 $[-10^9, 10^9]$ 範圍內的整數 $a, b, c$ 使得他們的和恰好為 $K$。
$$
	a + b + c = K
$$

\section*{Input}

第一行包含一個整數 $K$,代表欲構造的數字和。 

各變數範圍限制如下:
\begin{itemize}
	\item $-10^9 \le K \le 10^9$
\end{itemize}

\section*{Output}

請輸出三個整數,使得他們的和恰好為 $K$。
若有多組解,請輸出任何一組。

\section*{Sample 1}
\begin{longtable}[!h]{|p{0.5\textwidth}|p{0.5\textwidth}|}
\hline
\textbf {Input}	& \textbf {Output} \\
\hline
\parbox[t]{0.5\textwidth} % sample 1
{ \tt
% input
6 \\
} &
\parbox[t]{0.5\textwidth}
{ \tt
%output
1 2 3 \\
} \\
\hline
\end{longtable}

\section*{Sample 2}
\begin{longtable}[!h]{|p{0.5\textwidth}|p{0.5\textwidth}|}
\hline
\textbf {Input}	& \textbf {Output} \\
\hline
\parbox[t]{0.5\textwidth} % sample 2
{ \tt
% input
3 \\
} &
\parbox[t]{0.5\textwidth}
{ \tt
%output
1 1 1 \\
} \\
\hline
\end{longtable}

\section*{Sample 3}
\begin{longtable}[!h]{|p{0.5\textwidth}|p{0.5\textwidth}|}
\hline
\textbf {Input}	& \textbf {Output} \\
\hline
\parbox[t]{0.5\textwidth} % sample 3
{ \tt
% input
0 \\
} &
\parbox[t]{0.5\textwidth}
{ \tt
%output
-1 0 1 \\
} \\
\hline
\end{longtable}

\section*{配分}

在一個子任務的「測試資料範圍」的敘述中,如果存在沒有提到範圍的變數,則此變數的範圍為 Input 所描述的範圍。

\begin{center}
 \begin{tabular}{||c c c||} 
 \hline
 子任務編號 & 子任務配分 & 測試資料範圍 \\  
 \hline\hline
 0 & 0\% & 範例測資 \\ 
 \hline
 1 & 33\% & $K$ 是 $3$ 的倍數 \\
 \hline
 3 & 67\% & 無額外限制 \\
 \hline
\end{tabular}
\end{center}

\section*{Hint}
對於任何成立的 $a, b, c$,皆可以拿到分數。

\end{document}
