\documentclass[11pt,a4paper]{article}

\usepackage{indentfirst}
\usepackage{amssymb}
\usepackage{subcaption}
\usepackage{graphicx}
\usepackage{longtable}
\usepackage{fancyhdr}
\usepackage{xeCJK}
\usepackage{amsmath}
\usepackage{amssymb}
\usepackage{ulem}
\usepackage{xcolor}
\usepackage{fancyvrb}
\usepackage{listings}
\usepackage{soul}
\usepackage{hyperref}
	\lstdefinestyle{C}{
   		language=C, 
   		basicstyle=\ttfamily\bfseries,
    	numbers=left, 
    	numbersep=5pt,
    	tabsize=4,
    	frame=single,
   	 	commentstyle=\itshape\color{brown},
    	keywordstyle=\bfseries\color{blue},
   	 	deletekeywords={define},
    	morekeywords={NULL,bool}
	}	

\setCJKmainfont{Noto Sans Mono CJK TC}
 
\voffset -20pt
\textwidth 410pt
\textheight 650pt
\oddsidemargin 20pt
\newcommand{\XOR}{\otimes}
\linespread{1.2}\selectfont

\pagestyle{fancy}
\lhead{111 學年度宜蘭高中校內資訊學科能力競賽}

\begin{document}

\begin{center}
\section*{A. 成績批改}
\end{center}

\section*{Description}

據說,幾乎是在「考試」這個制度誕生的遙遠過去,「調分」這個概念早早就緊隨其後出現了,究竟是學生的實力不及還是老師的考卷出得太難,
這個世紀之謎到現在依舊還沒有正確的解答,這同時也成為所有老師的教育難題之一,要讓學生們滿意,究竟要如何調分,方法那是五花八門,
「線性調分」、「開根號乘十」,這些都是我們現在常聽到的幾種調分的方式,可以說各種調分方式都是老師們智慧的結晶,以台灣最常見的考試制度來說滿分為 $100$ 分,最低分為 $0$ 分,
不管怎麼調分,這都是不可撼動的上下界線。

回到現代,你是班上的數學小老師,你前往辦公室打算向數學老師拿班上的成績單,只是你發現數學老師早就坐在位置上與一疊厚厚的考卷戰鬥,
你是一個盡責的小老師,早就看出老師在調分了,於是你向老師詢問了調分方法後從厚厚的考卷堆裡拿走了 $N$ 份考卷,打算幫數學老師調分讓他早早去接他已經放學的小孩了。

調分方式很簡單,只需要把所有原始成績都加上 $C$ 分就好了,只不過需要注意,如果一名學生的原始成績加了 $C$ 分之後超過了 $100$ 分的話,根據不可撼動的考試制度,他的成績會被調整成剛好 $100$ 分。
	
\section*{Input}

第一行為兩個正整數 $N, C$,代表有幾 $N$ 份需要調分的考卷而每份考卷都需要向上調整 $C$ 分。

第二行為 $N$ 個正整數 $a_1 , a_2 \cdots a_N$,代表 $N$ 份考卷的原始成績。 

各變數範圍如下:
\begin{itemize}
    \item $1 \le N \le 1000$
    \item $1 \le C \le 100$
    \item $0 \le a_i \leq 100$
\end{itemize}\

\section*{Output}

請輸出 $N$ 個正整數,代表每份考卷調分後的分數。

\newpage
\section*{Sample 1}
\begin{longtable}[!h]{|p{0.5\textwidth}|p{0.5\textwidth}|}
\hline
\textbf {Input}	& \textbf {Output} \\
\hline
\parbox[t]{0.5\textwidth} % sample 1
{ \tt
% input
1 10\\
80\\
} &
\parbox[t]{0.5\textwidth}
{ \tt
%output
90\\
} \\
\hline
\end{longtable}

\section*{Sample 2}
\begin{longtable}[!h]{|p{0.5\textwidth}|p{0.5\textwidth}|}
\hline
\textbf {Input}	& \textbf {Output} \\
\hline
\parbox[t]{0.5\textwidth} % sample 2
{ \tt
% input
5 10\\
10 30 50 70 90\\
} &
\parbox[t]{0.5\textwidth}
{ \tt
%output
20 40 60 80 100\\
} \\
\hline
\end{longtable}

\section*{Sample 3}
\begin{longtable}[!h]{|p{0.5\textwidth}|p{0.5\textwidth}|}
\hline
\textbf {Input}	& \textbf {Output} \\
\hline
\parbox[t]{0.5\textwidth} % sample 3
{ \tt
% input
5 20\\
11 33 55 77 99\\
} &
\parbox[t]{0.5\textwidth}
{ \tt
%output
31 53 75 97 100\\
} \\
\hline
\end{longtable}

\section*{配分}

在一個子任務的「測試資料範圍」的敘述中,如果存在沒有提到範圍的變數,則此變數的範圍為 Input 所描述的範圍。

\begin{center}
 \begin{tabular}{||c c c||} 
 \hline
 子任務編號 & 子任務配分 & 測試資料範圍 \\  
 \hline
 \hline
 1 & 0\% & 範例測資 \\ 
 \hline
 2 & 11\% & $N =  1,\ \max(a_i) + C \le 100 $ \\
 \hline 
 3 & 31\% & $N = 1$ \\
 \hline
 4 & 31\% & $\max(a_i) + C \le 100$ \\
 \hline
 5 & 27\% & 無額外限制 \\
 \hline

\end{tabular}
\end{center}

\end{document}