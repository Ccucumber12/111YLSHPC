\documentclass[11pt,a4paper]{article}

\usepackage{indentfirst}
\usepackage{amssymb}
\usepackage{subcaption}
\usepackage{graphicx}
\usepackage{longtable}
\usepackage{fancyhdr}
\usepackage{xeCJK}
\usepackage{amsmath}
\usepackage{amssymb}
\usepackage{ulem}
\usepackage{xcolor}
\usepackage{fancyvrb}
\usepackage{listings}
\usepackage{soul}
\usepackage{hyperref}
	\lstdefinestyle{C}{
   		language=C, 
   		basicstyle=\ttfamily\bfseries,
    	numbers=left, 
    	numbersep=5pt,
    	tabsize=4,
    	frame=single,
   	 	commentstyle=\itshape\color{brown},
    	keywordstyle=\bfseries\color{blue},
   	 	deletekeywords={define},
    	morekeywords={NULL,bool}
	}	

\setCJKmainfont{Noto Sans Mono CJK TC}
 
\voffset -20pt
\textwidth 410pt
\textheight 650pt
\oddsidemargin 20pt
\newcommand{\XOR}{\otimes}
\linespread{1.2}\selectfont
\graphicspath{{images/}}

\pagestyle{fancy}
\lhead{國立宜蘭高中 111 學年度資訊學科能力競賽 - 測機}

\begin{document}

\begin{center}
\section*{B. Medium Task}
\end{center}

\section*{Description}

這是一題非常中等的測試題,你會拿到 $N$ 個數字,請計算這 $N$ 個數字的和。
$$
	a_1 + a_2 + \cdots + a_n = \sum_{i=1}^n a_i
$$

\section*{Input}

第一行包含一個正整數 $N$,代表欲相加的數字數量。 

第二行包含 $N$ 個整數 $a_i$,代表要相加的數字。

各變數範圍限制如下:
\begin{itemize}
	\item $1 \le N \le 10^6$ 
	\item $-10^9 \le a_i \le 10^9$ 
\end{itemize}

\section*{Output}

請輸出一個整數,代表 $N$ 個數字的和。

\section*{Sample 1}
\begin{longtable}[!h]{|p{0.5\textwidth}|p{0.5\textwidth}|}
\hline
\textbf {Input}	& \textbf {Output} \\
\hline
\parbox[t]{0.5\textwidth} % sample 1
{ \tt
% input
2 \\ 
11 22 \\
} &
\parbox[t]{0.5\textwidth}
{ \tt
%output
33 \\
} \\
\hline
\end{longtable}

\section*{Sample 2}
\begin{longtable}[!h]{|p{0.5\textwidth}|p{0.5\textwidth}|}
\hline
\textbf {Input}	& \textbf {Output} \\
\hline
\parbox[t]{0.5\textwidth} % sample 2
{ \tt
% input
5 \\
1 2 3 4 5 \\ 
} &
\parbox[t]{0.5\textwidth}
{ \tt
%output
15 \\
} \\
\hline
\end{longtable}

\newpage
\section*{Sample 3}
\begin{longtable}[!h]{|p{0.5\textwidth}|p{0.5\textwidth}|}
\hline
\textbf {Input}	& \textbf {Output} \\
\hline
\parbox[t]{0.5\textwidth} % sample 3
{ \tt
% input
1 \\
0 \\
} &
\parbox[t]{0.5\textwidth}
{ \tt
%output
0 \\
} \\
\hline
\end{longtable}

\section*{配分}

在一個子任務的「測試資料範圍」的敘述中,如果存在沒有提到範圍的變數,則此變數的範圍為 Input 所描述的範圍。

\begin{center}
 \begin{tabular}{||c c c||} 
 \hline
 子任務編號 & 子任務配分 & 測試資料範圍 \\  
 \hline\hline
 0 & 0\% & 範例測資 \\ 
 \hline
 1 & 15\% & $N = 2$ \\
 \hline
 2 & 60\% & $N \le 10^5$ \\
 \hline
 3 & 25\% & 無額外限制 \\
 \hline
\end{tabular}
\end{center}

\section*{Hint}
本題測試資料量大,建議使用 \tt{scanf} 進行輸入。
若使用 \tt{std::cin} 輸入,請在 \tt{main} 函式第一行加上 \tt{ios\_base::sync\_with\_stdio(0);} \tt{cin.tie(0);},且請勿跟 \tt{scanf} 混用,以免造成 Time Limit Exceeded。

\end{document}
